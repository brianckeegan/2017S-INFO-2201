 \documentclass[10pt]{memoir}

% based on kieran healy's memoir modifications
\usepackage{mako-mem}
\chapterstyle{article-2}
\pagestyle{mako-mem}

\usepackage{ucs}
\usepackage[utf8x]{inputenc}

%\usepackage{kpfonts}
%\usepackage[bitstream-charter]{mathdesign}
\usepackage{fbb}
\usepackage[T1]{fontenc}
%\usepackage{textcomp}

%\renewcommand{\rmdefault}{ugm}
%\renewcommand{\sfdefault}{phv}

% Packages for making a landscape table (?)
\usepackage[table,usenames,dvipsnames]{xcolor}
\usepackage{multirow}
\usepackage{pdflscape}
\usepackage{afterpage}

\usepackage[letterpaper,left=1.125in,right=1.125in,top=1.25in,bottom=1.25in]{geometry}

% packages i use in essentially every document
\usepackage{graphicx}
\usepackage{enumerate}

% Setup list environments
\usepackage{enumitem}
\setlist[description]{
  topsep=0pt,
  before=\vspace{0pt},
  after=\vspace{0pt},
  itemsep=2pt,
  labelsep=0pt
}

\setlist[itemize]{
    noitemsep, 
    leftmargin=1em,
    topsep=0pt}

% packages i use in many documents but leave off by default
% \usepackage{amsmath, amsthm, amssymb}
% \usepackage{dcolumn}
% \usepackage{endfloat}

% Set paragraph indents and spacing
\setlength{\parindent}{0pt}
\setlength{\parskip}{.5\baselineskip}
%\usepackage[document]{ragged2e}

% adjust section title formatting
\usepackage{titlesec}
\titlespacing\section{0pt}{8pt plus 2pt minus 2pt}{-6pt}
\titlespacing\subsection{0pt}{8pt plus 2pt minus 2pt}{-6pt}
\titlespacing\subsubsection{0pt}{8pt plus 2pt minus 2pt}{10pt}

% allows full, in-line citations
\usepackage{bibentry} 

% add bibliographic stuff 
\usepackage[round, numbers]{natbib} \def\citepos#1{\citeauthor{#1}'s (\citeyear{#1})} \def\citespos#1{\citeauthor{#1}' (\citeyear{#1})}
\renewcommand{\bibnumfmt}[1]{}

% define colors from http://www.colorado.edu/brand/visual-identity/typography-color
%\usepackage[usenames,dvipsnames]{color}
\definecolor{CUGold}{RGB}{207,184,124}
\definecolor{CUDarkGray}{RGB}{86,90,92}
\definecolor{CULightGray}{RGB}{162,164,163}

% customize URLs
\usepackage[hyphens]{url}
%\usepackage{breakurl} 
\usepackage[breaklinks, bookmarks, bookmarksopen]{hyperref}
\hypersetup{
    colorlinks=true,
    linkcolor=Blue,
    citecolor=Black,
    filecolor=Blue,
    urlcolor=Blue,
    unicode=true,
    breaklinks=true}

% create a "reading list" environment to format the items
\newenvironment{readinglist}{
\begin{list}{}{\leftmargin=0pt \itemindent=0em}
  \setlength{\itemsep}{8pt}
  \setlength{\parskip}{0em}
  \setlength{\parsep}{1em}
  \setlength{\parindent}{8em}}
{\end{list}}

% Course/Instructor metadata -- alter as neded
\def\myauthor{Brian Keegan}
\def\mycoursename{Computational Reasoning 2}
\def\mycourselisting{INFO 2201}
\def\myoffice{ENVD 201}
\def\myclassroom{Humanities 1B80}
\def\mymeetingtime{Monday, Wednesday, Friday 15:00--15:50}
\def\mydate{Spring 2017}
\def\myemail{brian.keegan@colorado.edu}
\def\myweb{http://www.brianckeegan.org}
\def\myofficehours{Monday 16:00--18:00 or by appointment}

% Some that I'm not using here:
\def\mytitle{Assistant Professor}
\def\mycopyright{\myauthor}
\def\myphone{(+1) 617-803-6971}
\def\mystreet{1060 18th St.}
\def\mycity{Boulder, CO 80302}

\begin{document}

\nobibliography*

%\baselineskip 14.2pt

\title{
    \textit{\normalsize{\textcolor{CUGold}{\textbf{\mycoursename:}}}}\\
    \textbf{\huge{Representations of Data}}\\
    \vspace{5pt} \normalsize{\mycourselisting}; \mydate
    }

\author{
Lecture: \mymeetingtime; \myclassroom\\
Lab 021: Monday, 13:00--13:50; Armory 201 \\
Lab 031: Monday, 14:00--14:50; Armory 201 \\ 
}

\date{\normalsize{\mytitle~\myauthor\\
       E-mail: \href{mailto:\myemail}{\myemail}\\
       Office: \myoffice\\
       Office hours: \myofficehours}}

\maketitle

%%%%%%%%%%%%%%%%%%%%%%
%% Acknowledgements %%
%%%%%%%%%%%%%%%%%%%%%%
% This syllabus template was made in LaTeX by Brian Keegan and is distributed as Free Software under the GNU GPL v3. It was built using style templates created by Aaron Shaw, Benjamin Mako Hill, and Kieran Healy.

\section{\textbf{Course Description}}

This course surveys techniques for representing data and expressing relationships among data. It will introduce fundamentals of algorithm analysis, review differences between fundamental data structures, implement classic algorithms, and cover methods for querying relational databases.

\subsection{Learning Objectives and Course Design}
The course consists of four modules. The first module on \textit{Fundamentals} focuses on developing familiarity with Python basics such as object-oriented programming, recursion, and algorithmic complexity. The second module on \textit{Structures} introduces several basic data structures used across the computer and information sciences as well as syntax for manipulating elementary data structures in Python. The third module on \textit{Algorithms} unit explores classic algorithms for manipulating and analyzing more advanced data structures like trees and graphs. The fourth module on \textit{Tables} covers the essentials for querying and manipulating tabular data. By the end of the semester, students will be able to evaluate the trade-offs in different basic data structures, implement classic algorithms for manipulating and analyzing data, and query tabular data using SQL syntax.

Class will meet three times per week on Monday, Wednesday, and Friday from 15:00 to 15:50 (3:00pm to 3:50pm) in \myclassroom~as well as a required laboratory section on Monday afternoon before class. The format of each class will vary between lectures and tutorials depending on the learning objectives. In addition to weekly participation and peer grading, students will complete two kinds of assignments: (1) weekly lab assignments and (2) Friday quizzes. There are no exams or final projects.

\subsection{Prerequisites}
Students should have completed an introductory course like INFO 1201, CSCI 1300/1310, or ATLS 1710 before enrolling in INFO 2201. If you have questions about these prerequisites, please \href{mailto:brian.keegan@colorado.edu}{email me}.

\subsection{Requirements}
Students' regular and sustained participation in all class activities as well as punctual and thorough completion of assignments are essential. If you need to be excused from attending a class session or need an extension to an assignment, please \href{mailto:brian.keegan@colorado.edu}{notify me via email} at least 24 hours in advance.

\subsection{Course Website and Materials}
There are two ``textbooks'' required for the class, both available via \url{http://computationaltales.blogspot.com/p/book.html#ComputationalFairyTales} (but not the CU Bookstore).

\begin{itemize}[itemsep=0em]
    \item \bibentry{kubica_computational_2012}
    \item \bibentry{kubica_detective_2016}
\end{itemize}

Additional readings, tutorials, and other material will be made available through \mbox{Desire2Learn}~(D2L):
\vspace{-8pt}
    \begin{center}
    \Large{\href{https://learn.colorado.edu/d2l/home/190526}{https://learn.colorado.edu/d2l/home/190526}}
    \end{center}
\vspace{-8pt}
Once the semester begins, this PDF version of the syllabus will be revised infrequently and any revised requirements will be posted as announcements and updated course schedule to D2L. The instructor reserves the right to make superseding changes to the course's schedule, evaluation criteria, policies, \textit{etc.} outlined below by making announcements via D2L, so please check D2L regularly.

\subsection{Statistical Computing}
You will need to use statistical computing software in this class. We will be using a combination of the  \href{https://www.jetbrains.com/pycharm-edu/download/}{PyCharm Edu} interactive development environment (IDE) and \href{http://jupyter.org/}{Jupyter notebooks} written in Python 3 for all in-class examples and lab assignments. (We may also get early access to a very exciting collaborative coding learning platform called \href{http://www.elice.io}{Elice} later in the term) I recommend using the \href{https://www.continuum.io/why-anaconda}{Anaconda distribution} of Python 3.5 (or above) to provide all of these programs and other libraries. We will discuss how to download and use the Anaconda package on your local machine during the first week of class. If students cannot bring a laptop to class, they should \href{mailto:brian.keegan@colorado.edu}{email me} to work out another arrangement.

\subsection{Evaluation} 
Your final grade for the course will be based on my evaluation of the three kinds of assignments as well as your general participation within discussions during class and online. 

    \begin{description}[itemsep=0pt,labelsep=0pt]
        \item[Lab Assignments]~($45\%$) Weekly Lab Assignments are intended to develop students' skill, intuition, and confidence writing Python code to understand the trade-offs between different data structures and algorithms. The Lab Assignments will primarily be drawn from the Kubica books assigned above. Each of these 15 weekly Lab Assignments will be worth 3\% of the final grade (45\% cumulative). Lab Assignments will be due on Wednesday before class and will be evaluated by a random anonymous peer on their completeness, use of good documentation practice, and clarity in addition to their correctness. Students should be prepared to explain their implementation in a 1-on-1 code review if I request it. \textit{In the absence of an approved excuse, late submissions will be docked 33\% of their value for every 24 hours elapsed since the deadline}: late assignments submitted after Saturday will lose all credit.
        \item[Peer Grading]~($15\%$) Peer Grading will expose students to each other's diverse problem-solving approaches and socialize students into supportive code reviewing practices. Each of the 15 weekly Lab Assignments will be graded anonymously by other students in class after submission. The primary foci of peer grading are (1) ensuring the assignment is complete and passes all unit tests, (2) ensuring that Lab Assignments follow good coding style and documentation practice, and (3) on providing feedback and suggestions about alternative approaches, unanticipated unit tests, \textit{etc}. Each on-time and substantive Peer Grade will be worth 1\% of the final grade (15\% cumulative). Peer Grades will be due on Friday before lecture and late Peer Grade submissions will forfeit this credit. To avoid abuse of this system, submitting students will be able to flag poor peer grading responses for my attention and I will also randomly spot-check submitted peer grades and deduct from the grader's Peer Grading credit if they fail to provide substantive feedback. Peer Graders will also be able to nominate creative Lab Assignments as well as identify unique unit tests for recognition during lecture.
        \item[Weekly Quizzes]~($24\%$) Students will complete an approximately 15 minute Weekly Quiz at the start of each Friday lecture. These quizzes will provide important and timely feedback for both the instructor and the students about the difficulty of materials covered in the previous two sessions. Each quiz will be worth 2\% of the final grade but only the 12 highest Weekly Quiz scores will be calculated in the final grade, meaning students can miss three quizzes with no penalty. \textit{There will be absolutely no make-up opportunities for missed quizzes beyond these ``three strikes''}.  %Each quiz will be weighted by the standard score in cumulative snowfall for the preceding Thursday--Thursday week as measured by \href{http://www.onthesnow.com/colorado/breckenridge/historical-snowfall.html}{OnTheSnow.com} and compared to \href{http://www.wrcc.dri.edu/cgi-bin/cliMAIN.pl?co0909}{historical records for Breckenridge}: quizzes will be worth more if there was more snowfall than average in that week and vice versa. 
        \item[Participation]~($16\%$) Participation will be evaluated in combination with attendance. Students' participation across the three lectures and lab section each week will be worth approximately 1\% of the final grade (16\% cumulative). Perfect attendance without engagement will be penalized as much or more than infrequent attendance with active engagement. Students demonstrating consistent attendance, high preparation, positive attitude, novel perspectives, and persuasive arguments will earn the most credit. If you have a disability, anxiety, or other issues that limit your ability to actively participate in a lecture format, please contact me to discuss alternative participation methods. \textit{In the absence of an approved excuse, there will be no opportunities to make up for missed participation.}
    \end{description}
    
\section{Weekly Routine}
The class has several moving pieces spread across multiple days for each topic, but will follow a regular routine throughout the term that students should anticipate. I have adapted this framework from Professor Ricarose Roque's course design for INFO 1201. Each week will introduce a new concept in lecture on Monday and dive deeper on this concept during Wednesday's lecture. The Friday lecture will include a quiz on those concepts, a discussion about difficulties with the concepts, and then introduce the lab assignment. Students will work on the lab assignment over the weekend and during the following Monday lab sessions before their following Wednesday deadline. Students will be receive their Peer Grading assignment after lecture on that Wednesday and will submit the Peer Grading score on Friday before lecture.

\begin{table}[h]
    \scriptsize
    \centering
    \begin{tabular}{ p{1.7cm} p{3.9cm} p{3.9cm} p{3.9cm} }
        \toprule[.15em]
        \multicolumn{1}{c}{\textbf{Activity}} & \multicolumn{1}{c}{\textbf{Monday}} & \multicolumn{1}{c}{\textbf{Wednesday}} & \multicolumn{1}{c}{\textbf{Friday}} \\
        \cmidrule[.1em](lr){1-4}
        \textit{Theme} & \multicolumn{1}{c}{Introduction} & \multicolumn{1}{c}{Re-inforcement} & \multicolumn{1}{c}{Reflection} \\
        \cmidrule[.1em](lr){1-4}
        \textit{Lecture} & Outline weekly goals, introduce new concepts, motivate their relevance & Reinforce concepts with additional details, working through applications & Quiz and discussion followed by weekly lab assignment introduction \\ 
        \cmidrule[.1em](lr){1-4}
        \textit{Lab Assignment} & Use sections to finish previous Lab Assignment and answer questions & Previous Lab Assignment is due before lecture & Next Lab Assignment and its grading rubric introduced during lecture \\ 
        \cmidrule[.1em](lr){1-4}
        \textit{Peer Grading} & Nothing & Receive peer's previous Lab Assignment after lecture, begin peer grading & Submit scores and nominate solutions/unit tests before lecture \\ 
        \cmidrule[.1em](lr){1-4}
        \textit{Weekly Quiz} & Nothing & Prepare with example problems in supporting materials & Quiz at start of class; reflection and feedback in lecture \\
        \bottomrule[.15em]
    \end{tabular}
\end{table}


\section{Course Policies}

\subsection{In-Class Confidentiality}
The success of this class depends on participants feeling comfortable sharing questions, ideas, concerns, and confusions about lab assignments, peer grading, research or entrepreneurial work-in-progress, and their personal experiences. These assignments and discussions should be considered confidential and generally should not be discussed outside of class. You may read, comment, and run on classmates' writing, code, and other class-related content for the sole purpose of use within this class. However, you may not use, run, copy, perform, display, distribute, modify, translate, or create derivative works of other students' work outside of this class without their expressed written consent or formal license. Furthermore, you may not create any audio, video, or other records during class time nor may you publicly share comments made in class attributable to another person's identity without that person's permission.

%\subsection{Peer Grading}
%Programming is a fundamentally creative and collaborative endeavor. We will experiment with at least one peer grading model where students evaluate each other's lab assignments. For each week's Lab Assignment, each student will grade a randomly selected colleague's Lab Assignment following a rubric I will distribute. 

\subsection{Faculty Interaction}
In addition to teaching this class, I also (1) manage a research program; (2) advise students; (3) perform service for the academic community; and (4) live my life as a private citizen. Interestingly, my middle name is Christopher, my favorite band is \href{http://www.fightoffyourdemons.com/}{Brand New}, I played the cello in high school, and my first job after college was bartending. I will check e-mail between 8:00 and 18:00 on non-holiday business days and try to respond within 24 hours. I welcome online or offline interactions outside of class, however these are not appropriate spaces for discussing class matters. \href{maito:brian.keegan@colorado.edu}{E-mailing me} or coming to my office hours are the best ways to get feedback outside of class.

%\subsection{Deadlines and Absences}
%If something causes you to miss a deadline or a class, please contact me. If you request --- and obtain --- an incomplete for the course and/or an extension on the final project (note: I \textit{strongly} discourage this!), please allow at least 4 weeks after you submit your completed work for me to submit a grade. Keep this in mind if you will need the grade to receive your fellowship/diploma/visa/\textit{etc}. by a particular date.

\subsection{Accommodations for Disabilities}
I am committed to providing everyone the support and services needed to participate in this course. If you qualify for accommodations because of a disability, please submit to your professor a letter from Disability Services in a timely manner (for exam accommodations provide your letter at least one week prior to the exam) so that your needs can be addressed. Disability Services determines accommodations based on documented disabilities. Contact Disability Services at 303-492-8671 or by e-mail at \href{mailto:dsinfo@colorado.edu}{dsinfo@colorado.edu}. If you have a temporary medical condition or injury, see \href{http://www.colorado.edu/disabilityservices/students/temporary-medical-conditions}{Temporary Medical Conditions} guidelines under Quick Links at \href{http://www.colorado.edu/disabilityservices/students}{Disability Services} website and discuss your needs with me.

\subsection{Religious Observance}
Campus policy regarding \href{http://www.colorado.edu/policies/observance-religious-holidays-and-absences-classes-andor-exams}{religious observances} requires that faculty make every effort to deal reasonably and fairly with all students who, because of religious obligations, have conflicts with scheduled exams, assignments or required assignments/attendance. If this applies to you, please email me directly as soon as possible at the beginning of the term. 

\subsection{Classroom Behavior}
Students and faculty each have responsibility for maintaining an appropriate learning environment. Those who fail to adhere to such behavioral standards may be subject to discipline. Professional courtesy and sensitivity are especially important with respect to individuals and topics dealing with differences of race, color, culture, religion, creed, politics, veteran’s status, sexual orientation, gender, gender identity and gender expression, age, ability, and nationality.  Class rosters are provided to the instructor with the student's legal name. I will gladly honor your request to address you by an alternate name or gender pronoun. Please advise me of this preference early in the semester so that I may make appropriate changes. For more information, see the policies on \href{http://www.colorado.edu/policies/student-classroom-and-course-related-behavior}{class behavior} and the \href{http://www.colorado.edu/osc/#student_code}{student code}.

\subsection{Harassment and Discrimination}
The University of Colorado Boulder (CU Boulder) is committed to maintaining a positive learning, working, and living environment. CU Boulder will not tolerate acts of sexual misconduct, discrimination, harassment or related retaliation against or by any employee or student. CU's \href{http://www.colorado.edu/policies/discrimination-and-harassment-policy-and-procedures}{Sexual Misconduct Policy} prohibits sexual assault, sexual exploitation, sexual harassment, intimate partner abuse (dating or domestic violence), stalking or related retaliation. CU Boulder's \href{http://www.colorado.edu/policies/discrimination-and-harassment-policy-and-procedures}{Discrimination and Harassment Policy} prohibits discrimination, harassment or related retaliation based on race, color, national origin, sex, pregnancy, age, disability, creed, religion, sexual orientation, gender identity, gender expression, veteran status, political affiliation or political philosophy. Individuals who believe they have been subject to misconduct under either policy should contact the Office of Institutional Equity and Compliance (OIEC) at 303-492-2127. Information about the OIEC, the above referenced policies, and the campus resources available to assist individuals regarding sexual misconduct, discrimination, harassment or related retaliation can be found at the \href{http://www.colorado.edu/institutionalequity/}{OIEC website}.

\subsection{Honor Code}
All students enrolled in a University of Colorado Boulder course are responsible for knowing and adhering to the \href{http://www.colorado.edu/policies/academic-integrity-policy}{academic integrity policy} of the institution. Violations of the policy may include: plagiarism, cheating, fabrication, lying, bribery, threat, unauthorized access, clicker fraud, resubmission, and aiding academic dishonesty. All incidents of academic misconduct will be reported to the Honor Code Council (\href{mailto:honor@colorado.edu}{honor@colorado.edu}; 303-735-2273). Students who are found responsible for violating the academic integrity policy will be subject to nonacademic sanctions from the Honor Code Council as well as academic sanctions from the faculty member. Additional information can be found at \href{http://honorcode.colorado.edu}{honorcode.colorado.edu}. 

\section{Acknowledgements}
The design and format of this course borrows from similar courses and textbooks. You are not required to buy these books or classes, but these can be very helpful resources if you find yourself struggling.
\begin{itemize}[itemsep=1em]
    \item Textbooks
        \begin{itemize}
        \item \bibentry{bhargava_grokking_2016}
        \item \bibentry{goodrich_data_2013}
        \item \bibentry{guttag_introduction_2016}
        \item \bibentry{heineman_algorithms_2016}
        \item \bibentry{hetland_python_2014}
        \item \bibentry{karumanchi_data_2015}
        \item \bibentry{lambert_fundamentals_2014}
        \item \bibentry{lee_data_2015}
        \item \bibentry{miller_problem_2013}
        \item \bibentry{pilgrim_dive_2009}
        \item \bibentry{stephens_essential_2013}
        \end{itemize}
    \item Online courses
        \begin{itemize}
        \item \bibentry{balazs_algorithms_2016}
        \item \bibentry{kane_algorithmic_2016}
        \item \bibentry{nakhleh_algorithmic1_2016}
        \item \bibentry{portilla_python_2016}
        \item \bibentry{roughgarden_online_2016}
        \item \bibentry{sentance_python_2016}
        \end{itemize}
\end{itemize}

This syllabus was typeset in \LaTeX~using \href{http://www.sharelatex.com}{ShareLaTeX} with the \href{http://www.tug.dk/FontCatalogue/fbb/}{fbb/Bembo} font and is derived from the \texttt{memoir} styles adapted by \href{https://github.com/kjhealy/latex-custom-kjh}{Kieran Healy} and \href{http://projects.mako.cc/source/?p=latex_mako;a=summary}{Benjamin `Mako' Hill}.

%%%%%%%%%%%%%%%%%%%%%%
%%%%%%%%%%%%%%%%%%%%%%
%%% COURSE OUTLINE %%%
%%%%%%%%%%%%%%%%%%%%%%
%%%%%%%%%%%%%%%%%%%%%%

\newpage
\section{\textbf{Course Outline}}

I may adjust the list of readings or the schedule as needed throughout the semester, so please consult the schedule online at Desire2Learn for the most up-to-date information. We will meet in class a total of 44 times over the course of the the 16-week semester.

% Confirm research computing environments for Python and \texttt{R} are properly configured and ready for data collection and analysis. 

\section{Week 1 -- Fundamentals: Review}
\textcolor{CUGold}{\textbf{Wednesday, January 18; Friday, January 20}}\\
Configure computing environment and review concepts from INFO 1201. % operations on fundamental Python data structures.
    
    \subsection{Objectives}
    \begin{itemize}
        \item Setup and introduce PyCharm Edu and Jupyter Notebook environments.
        \item Review concepts like loops, conditions, and functions.
        \item Introduce documentation standards and peer grading standards.
    \end{itemize}
    
    \subsection{Assignments}
    \begin{description}
        \item[Lab 1 ] -- due Wednesday, January 25. 
    \end{description}

\section{Week 2 -- Fundamentals: Objects}
\textcolor{CUGold}{\textbf{Monday, January 23; Wednesday, January 25; Friday, January 27}}\\
Introduction to object-oriented programming within Python using classes.
    
    \subsection{Objectives}
    \begin{itemize}
        \item Initializing and interacting with classes.
        \item Defining functions, attributes, and methods.
        \item Inheritance of properties across objects.
    \end{itemize}

    \subsection{Assignments}
    \begin{description}
        \item[Lab 2 ] -- due Wednesday, February 1 % Metamorphosing Data
    \end{description}

\section{Week 3 -- Fundamentals: Recursion}
\textcolor{CUGold}{\textbf{Monday, January 30; Wednesday, February 1; Friday, February 3}}\\
Developing intuitions with recursive functions and identifying cases to apply them.

    \subsection{Objectives}
    \begin{itemize}
        \item Defining functions calling themselves.
        \item Identifying base and recursive cases.
        \item Pushing and popping items from a call stack.
    \end{itemize}

    \subsection{Assignments}
    \begin{description}
        \item[Lab 3 ] -- due Wednesday, February 8.
    \end{description}

\section{Week 4 -- Fundamentals: Complexity}
\textcolor{CUGold}{\textbf{Monday, February 6; Wednesday, February 8; Friday, February 10}}\\
Measuring the time complexity of functions and reviewing efficiency of different design patterns.

    \subsection{Objectives}
    \begin{itemize}
        \item Demonstrate importance of time complexity and asymptotic analysis.
        \item Review Big-O notation and complexity classes.
        \item Identify primitive operations affecting algorithm complexity.
    \end{itemize}

    \subsection{Assignments}
    \begin{description}
        \item[Lab 4 ] -- due Wednesday, February 15. % Psychadelic art
    \end{description}

\section{Week 5 -- Structures: Linear}
\textcolor{CUGold}{\textbf{Monday, February 13; Wednesday, February 15; Friday, February 17}}\\
Introduction to linear data structures like arrays, matrices, and linked lists.

    \subsection{Objectives}
    \begin{itemize}
        \item Comparing 1-dimensional arrays, linked lists, and matrices.
        \item Inserting and deleting values in linear data structures.
        \item Indexing and swapping values in linear data structures.
    \end{itemize}

    \subsection{Assignments}
    \begin{description}
        \item[Lab 5 ] -- due Wednesday, February 22.
    \end{description}

\section{Week 6 -- Structures: Unique}
\textcolor{CUGold}{\textbf{Monday, February 20; Wednesday, February 22; Friday, February 24}}\\
Introduction to data collections where values are indexed by unique keys.

    \subsection{Objectives}
    \begin{itemize}
        \item Operations, applications, and limitations of sets and dictionaries.
        \item Implementing hashing functions to generate key-value pairs.
        \item Comparing array and hashing implementations of dictionaries. 
    \end{itemize}

    \subsection{Assignments}
    \begin{description}
        \item[Lab 6 ] -- due Wednesday, March 1. % Evaluating Chess Moves
    \end{description}
    
\section{Week 7 -- Structures: Trees}
\textcolor{CUGold}{\textbf{Monday, February 27; Wednesday, March 1; Friday, March 3}}\\
Accessing and analyzing hierarchical properties of trees. 

\textbf{NOTE:} \textit{Professor Keegan will be out-of-town attending the \href{https://cscw.acm.org/2017/}{CSCW 2017} conference Monday and Wednesday. There will be a guest to cover the Monday and Wednesday lectures and labs.}

    \subsection{Objectives}
    \begin{itemize}
        \item Implementing and altering trees with basic data structures.
        \item Methods for traversing and manipulating trees.
        \item Comparing depth- and breadth-first search algorithms.
    \end{itemize}

    \subsection{Assignments}
    \begin{description}
        \item[Lab 7 ] -- due Wednesday, March 8.
    \end{description}
    
\section{Week 8 -- Algorithms: Strings}
\textcolor{CUGold}{\textbf{Monday, March 6; Wednesday, March 8; Friday, March 10}}\\
Using algorithms to manipulate, compare, and find strings.

    \subsection{Objectives}
    \begin{itemize}
        \item Implementing algorithms to manipulate and format strings.
        \item Using algorithms to measure string distance and similarity.
        \item Fundamentals and limitations of regular expressions for finding sub-strings.
    \end{itemize}

    \subsection{Assignments}
    \begin{description}
        \item[Lab 8 ] -- due Wednesday, March 15.
    \end{description}
    
\section{Week 9 -- Algorithms: Sorting}
\textcolor{CUGold}{\textbf{Monday, March 13; Wednesday, March 15; Friday, March 17}}\\
Using algorithms to efficiently order values.

    \subsection{Objectives}
    \begin{itemize}
        \item Implementing and comparing exchange, selection, insertion, and merge sorting.
    \end{itemize}

    \subsection{Assignments}
    \begin{description}
        \item[Lab 9 ] -- due Wednesday, March 22.
    \end{description}
    
\section{Week 10 -- Algorithms: Graphs}
\textcolor{CUGold}{\textbf{Monday, March 20; Wednesday, March 22; Friday, March 24}}\\
Accessing and analyzing structural properties of graphs.

    \subsection{Objectives}
    \begin{itemize}
        \item Types and representations of graphs.
        \item Algorithms for graph traversal and manipulation.
        \item Computing basic network science metrics.
    \end{itemize}

    \subsection{Assignments}
    \begin{description}
        \item[Lab 10 ] -- due Wednesday, April 5.
    \end{description}
    
\section{Week 11 -- Spring Break}
\textcolor{CUGold}{\textbf{Monday, March 27; Wednesday, March 29; Friday, March 31}}\\
Adding new objects into long-term memory.

\section{Week 12 -- Databases: \texttt{pandas}}
\textcolor{CUGold}{\textbf{Monday, April 3; Wednesday, April 5; Friday, April 7}}\\
Using a panel data library to reshape, manipulate, and visualize tabular data.

    \subsection{Objectives}
    \begin{itemize}
        \item Loading data from CSV files into DataFrames.
        \item Converting between data shapes by melting, pivoting, and stacking.
        \item Visualizing tabular data with \texttt{matplotlib} and \texttt{seaborn}.
    \end{itemize}

    \subsection{Assignments}
    \begin{description}
        \item[Lab 12 ] -- due Wednesday, April 12.
    \end{description}
    
\section{Week 13 -- Databases: Selecting}
\textcolor{CUGold}{\textbf{Monday, April 10; Wednesday April 12; Friday, April 14}}\\
Applying basic SQL syntax for selecting, filtering, sorting, and counting data.

    \subsection{Objectives}
    \begin{itemize}
        \item Selecting columns and sorting results.
        \item Filtering results with \texttt{where} syntax.
        \item Summarizing results with functions and aliases.
    \end{itemize}

    \subsection{Assignments}
    \begin{description}
        \item[Lab 13 ] -- due Wednesday, April 19.
    \end{description}
    
\section{Week 14 -- Databases: Joining}
\textcolor{CUGold}{\textbf{Monday, April 17; Wednesday, April 19; Friday, April 21}}\\
Combining data from two different tables and dealing with missing or repeated data. 

    \subsection{Objectives}
    \begin{itemize}
        \item Joining on single and multiple keys.
        \item Comparing left, right, inner, and outer joins.
        \item Managing joins of multiple tables and duplicate columns.
    \end{itemize}

    \subsection{Assignments}
    \begin{description}
        \item[Lab 14 ] -- due Wednesday, April 26.
    \end{description}
    
\section{Week 15 -- Databases: Grouping}
\textcolor{CUGold}{\textbf{Monday, April 24; Wednesday, April 26; Friday, April 28}}\\
Dealing with repeated observations and computing aggregated values.

    \subsection{Objectives}
    \begin{itemize}
        \item Identifying and grouping repeated data values.
        \item Specifying aggregation functions to summarize grouped values.
        \item Using having clause to subset group by results.
    \end{itemize}

    \subsection{Assignments}
    \begin{description}
        \item[Lab 15 ] -- due Wednesday, May 3.
    \end{description}
    
\section{Week 16 -- Databases: Exotic}
\textcolor{CUGold}{\textbf{Monday, May 1; Wednesday, May 3; Friday, May 5}}\\
Relational databases alternatives like NoSQL and graph databases as well as map-reduce frameworks.

    \subsection{Objectives}
    \begin{itemize}
        \item Storing and querying JSON data in a NoSQL database like MongoDB.
        \item Querying complex networks in a graph database like Neo4j using Cypher.
        \item Parallelizing data mining with map-reduce frameworks like Hadoop.
    \end{itemize}

    \subsection{Assignments}
    \begin{description}
        \item[Lab 16 ] -- due Wednesday, May 10.
    \end{description}


% Searching & linear, binary, interpolation
% Sequential & sequences, iterators, generators
% Primitives & lists, dictionaries, tuples, sets, strings
% \texttt{numpy} & ndarrays, manipulation, computation \\ 

\afterpage{%
    \clearpage
    \thispagestyle{empty}
    \begin{landscape}
        \centering
        \Large
        \begin{tabular}{ccccc}
            \toprule[.15em]
            \textbf{Module} & \textbf{Week} & \textbf{Dates} & \textbf{Theme} & \textbf{Topics} \\
            \cmidrule[.1em](lr){1-5}
            \multirow{4}{*}[0pt]{\textit{Fundamentals}}
                & 1 & Jan 18 -- Jan 20 & Review & loops, conditions, functions \\
                & 2 & Jan 23 -- Jan 27 & Objects & classes, namespaces, inheritance \\
                & 3 & Jan 30 -- Feb 3 & Recursion & base case, head/tail, stack diagram \\
                & 4 & Feb 6 -- Feb 10 & Complexity & Big-O, asymptotic behavior \\ \cmidrule[.1em](lr){1-5}
            
            \multirow{3}{*}[0pt]{\textit{Structures}}     
                & 5 & Feb 13 -- Feb 17 & Linear & arrays, linked lists, matrices \\
                & 6 & Feb 20 -- Feb 24 & Unique & sets, hashes, dictionaries \\
                & 7 & Feb 27 -- Mar 3 & Trees & construction, traversal \\ \cmidrule[.1em](lr){1-5}
                
            \multirow{3}{*}[0pt]{\textit{Algorithms}} 
                & 8 & Mar 6 -- Mar 10 & Strings & similarity, searching, matching \\
                & 9 & Mar 13 -- Mar 17 & Sorting & insertion, bubble, merge \\
                & 10 & Mar 20 -- Mar 24 & Graphs & formats, search, connectivity \\
                
                \cmidrule[.1em](lr){1-5}
                & 11 & Mar 27 -- Mar 31 & \multicolumn{2}{c}{Spring Break} \\
                \cmidrule[.1em](lr){1-5}
                
            \multirow{5}{*}[0pt]{\textit{Tables}} 
                & 12 & Apr 3 -- Apr 7 & \texttt{pandas} & dataframes, indexing, plotting \\
                & 13 & Apr 10 -- Apr 14 & Selecting & aliases, order, where, functions \\
                & 14 & Apr 17 -- Apr 21 & Joining & keys, inner, outer, left, right \\
                & 15 & Apr 24 -- Apr 28 & Grouping & groupby, aggregation, having  \\
                & 16 & May 1 -- May 5 & Exotic & NoSQL, graph, map-reduce \\
            \bottomrule[.15em]
        \end{tabular}\\
        %\caption{Course Schedule}% Add 'table' caption
    \end{landscape}
    \clearpage% Flush page
}

% bibliography here
%\newpage
\renewcommand{\bibsection}{\section{\huge \bibname}\prebibhook}
\baselineskip 14.2pt
\nobibliography{refs}
\bibliographystyle{apalike}

\end{document}
