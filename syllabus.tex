\documentclass[10pt]{memoir}

% based on kieran healy's memoir modifications
\usepackage{mako-mem}
\chapterstyle{article-2}
\pagestyle{mako-mem}

\usepackage{ucs}
\usepackage[utf8x]{inputenc}

%\usepackage{kpfonts}
%\usepackage[bitstream-charter]{mathdesign}
\usepackage{fbb}
\usepackage[T1]{fontenc}
%\usepackage{textcomp}

%\renewcommand{\rmdefault}{ugm}
%\renewcommand{\sfdefault}{phv}

% Packages for making a landscape table (?)
\usepackage[table,usenames,dvipsnames]{xcolor}
\usepackage{multirow}
\usepackage{pdflscape}
\usepackage{afterpage}

\usepackage[letterpaper,left=1.25in,right=1.25in,top=1.25in,bottom=1.25in]{geometry}

% packages i use in essentially every document
\usepackage{graphicx}
\usepackage{enumerate}

% Setup list environments
\usepackage{enumitem}
\setlist[description]{
  topsep=0pt,
  before=\vspace{0pt},
  after=\vspace{0pt},
  itemsep=2pt,
  labelsep=0pt
}

\setlist[itemize]{
    noitemsep, 
    leftmargin=1em,
    topsep=0pt}

% packages i use in many documents but leave off by default
% \usepackage{amsmath, amsthm, amssymb}
% \usepackage{dcolumn}
% \usepackage{endfloat}

% Set paragraph indents and spacing
\setlength{\parindent}{0pt}
\setlength{\parskip}{.5\baselineskip}
%\usepackage[document]{ragged2e}

% adjust section title formatting
\usepackage{titlesec}
\titlespacing\section{0pt}{8pt plus 2pt minus 2pt}{-6pt}
\titlespacing\subsection{0pt}{8pt plus 2pt minus 2pt}{-6pt}
\titlespacing\subsubsection{0pt}{8pt plus 2pt minus 2pt}{10pt}

% allows full, in-line citations
\usepackage{bibentry} 

% add bibliographic stuff 
\usepackage[round, numbers]{natbib} \def\citepos#1{\citeauthor{#1}'s (\citeyear{#1})} \def\citespos#1{\citeauthor{#1}' (\citeyear{#1})}
\renewcommand{\bibnumfmt}[1]{}

% define colors from http://www.colorado.edu/brand/visual-identity/typography-color
%\usepackage[usenames,dvipsnames]{color}
\definecolor{CUGold}{RGB}{207,184,124}
\definecolor{CUDarkGray}{RGB}{86,90,92}
\definecolor{CULightGray}{RGB}{162,164,163}

% customize URLs
\usepackage[hyphens]{url}
%\usepackage{breakurl} 
\usepackage[breaklinks, bookmarks, bookmarksopen]{hyperref}
\hypersetup{
    colorlinks=true,
    linkcolor=Blue,
    citecolor=Black,
    filecolor=Blue,
    urlcolor=Blue,
    unicode=true,
    breaklinks=true}

% create a "reading list" environment to format the items
\newenvironment{readinglist}{
\begin{list}{}{\leftmargin=0pt \itemindent=0em}
  \setlength{\itemsep}{8pt}
  \setlength{\parskip}{0em}
  \setlength{\parsep}{1em}
  \setlength{\parindent}{8em}}
{\end{list}}

% Course/Instructor metadata -- alter as neded
\def\myauthor{Brian Keegan}
\def\mycoursename{Computational Reasoning 2}
\def\mycourselisting{INFO 2201}
\def\myoffice{ENVD 201}
\def\myclassroom{Humanities 1B80}
\def\mymeetingtime{Monday, Wednesday, Friday 15:00--16:00}
\def\mydate{Spring 2017}
\def\myemail{brian.keegan@colorado.edu}
\def\myweb{http://www.brianckeegan.org}
\def\myofficehours{Monday 16:00--18:00 or by appointment}

% Some that I'm not using here:
\def\mytitle{Assistant Professor}
\def\mycopyright{\myauthor}
\def\myphone{(+1) 617-803-6971}
\def\mystreet{1060 18th St.}
\def\mycity{Boulder, CO 80302}

\begin{document}

\nobibliography*

%\baselineskip 14.2pt

\title{
    \textit{\normalsize{\textcolor{CUGold}{\textbf{\mycoursename:}}}}\\
    \textbf{\huge{Representations of Data}}\\
    \vspace{5pt} \normalsize{\mycourselisting}
    }

\author{\mydate\\ \mymeetingtime\\ \myclassroom}

\date{\normalsize{\mytitle~\myauthor\\
       E-mail: \href{mailto:\myemail}{\myemail}\\
       Office: \myoffice\\
       Office hours: \myofficehours}}

\maketitle

%%%%%%%%%%%%%%%%%%%%%%
%% Acknowledgements %%
%%%%%%%%%%%%%%%%%%%%%%
% This syllabus template was made in LaTeX by Brian Keegan and is distributed as Free Software under the GNU GPL v3. It was built using style templates created by Aaron Shaw, Benjamin Mako Hill, and Kieran Healy.

\section{\textbf{Course Description}}

%This is a research seminar that will analyze the social and technical mechanisms that enable popular peer production and crowdsourcing systems like Wikipedia. Understanding how social processes and technical structures intersect to enable new kinds of interaction is a central question within information science. How does the design of a peer production system tap into basic social, psychological, and organizational processes? What kinds of behavioral data can researcher extract and analyze from these systems? How can the success or failure of these systems inform the design of alternative economic and political models? This course is part of the ``Problems in Information Science'' series, which brings contemporary research to the classroom in the form of project-based investigation.

Surveys techniques for representing data and expressing relationships among data, both at small scales (for example, via programmatic data structures) and at large scales (for example, in various kinds of database systems). Introduces fundamentals of algorithm analysis and the trade-offs involved in managing data using different approaches, tools and organizing principles.


\subsection{Learning Objectives and Course Design}
By the end of the semester, students will be able to: (1) evaluate the trade-offs in different basic data structures, (2) implement classic algorithms for manipulating and analyzing data across structures, (3) reshape data into different formats, and (4) query databases using SQL syntax.

%(1) discuss and compare social processes occurring across different peer production and crowdsourcing systems; (2) identify and collect data generated by these systems; (3) design exploratory research using these data and interpret their findings; and (4) propose and evaluate the feasibility of peer production and crowdsourcing designs to expand to other social, cultural, economic, and political spheres.

The course consists of four units. The first unit focuses on several basic data structures used across the computer and information sciences as well as syntax for manipulating elementary data structures in Python. The second unit explores classic algorithms for manipulating and analyzing more advanced data structures like trees and graphs. The third unit compares different approaches to shaping data from flat arrays to more complex hierarchical indices and markup languages. The final unit will introduce students to the syntax for querying relational and non-relational databases.

Class will meet three times per week on Monday, Wednesday, and Friday from 15:00 to 16:00 (3:00pm to 4:00pm) in \myclassroom~as well as a laboratory section on Monday afternoon before class. The format of each class will vary between lectures and tutorials depending on the learning objectives of that week. Students will complete two kinds of assignments over the course of the class: (1) lab assignments and (2) weekly quizzes.

\subsection{Prerequisites}
Students should have completed an introductory computing course like INFO 1201, CSCI 1300/1310, or ATLS 1710 before enrolling in INFO 2201. If you have questions or concerns about these prerequisites, please \href{mailto:brian.keegan@colorado.edu}{email me}.

\subsection{Requirements}
Students' regular and sustained participation in all class activities as well as punctual and thorough completion of assignments are essential. If you need to be excused from attending a class session or need an extension to an assignment, please \href{mailto:brian.keegan@colorado.edu}{notify me via email} at least 24 hours in advance.

\subsection{Course Website and Materials}
There is no textbook for the class but slides, readings, tutorials, lab assignments, and other material will be made available through \mbox{Desire2Learn}~(D2L):
\vspace{-8pt}
    \begin{center}
    \Large{\href{https://learn.colorado.edu/d2l/home/190526}{https://learn.colorado.edu/d2l/home/190526}}
    \end{center}
\vspace{-8pt}
Once the semester begins, this PDF version of the syllabus will be revised infrequently and any revised requirements will be posted as announcements and updated course schedule to D2L.

\subsection{Statistical Computing}
You will need to use statistical computing software in this class. \href{http://jupyter.org/}{Jupyter notebooks} written in Python will be used for all in-class examples and lab assignments. I recommend using the \href{https://www.continuum.io/why-anaconda}{Anaconda distribution} of Python for this class.

\subsection{Evaluation} 
Your final grade for the course will be based on my evaluation of the three kinds of assignments as well as your general participation within discussions during class and online. 

    \begin{description}[itemsep=0pt,labelsep=0pt]
        \item[Lab Assignments]~($60\%$) Lab assignments are intended to develop students' skill and confidence performing exploratory data analyses using real world data from peer production systems. Students should include their well-commented code documenting their process as well as the final outputs (\textit{e.g.}, submitting the Jupyter Notebook file). Students should be prepared to explain their implementation in a 1-on-1 code review if I request it. Assignments will be primarily evaluated on their completeness, use of good documentation practice, and clarity when discussing results.
        \item[Friday quizzes]~($25\%$) Students will complete a 10-15 minute quiz at the start of each Friday class meeting. These quizzes will provide important and timely feedback for both the instructor and the students about the difficulty of materials covered in the previous two sessions. Each quiz will be worth 2\% of the final grade (30\% cumulative), meaning students can miss $\sim$3 quizzes with no penalty --- or alternatively take every quiz for extra credit. \textit{There will be absolutely no make-up opportunities for missed quizzes beyond these ``three strikes''}.  %Each quiz will be weighted by the standard score in cumulative snowfall for the preceding Thursday--Thursday week as measured by \href{http://www.onthesnow.com/colorado/breckenridge/historical-snowfall.html}{OnTheSnow.com} and compared to \href{http://www.wrcc.dri.edu/cgi-bin/cliMAIN.pl?co0909}{historical records for Breckenridge}: quizzes will be worth more if there was more snowfall than average in that week and vice versa. 
        \item[Participation]~($15\%$) Students demonstrating consistent attendance, active engagement, high preparation, deep curiosity, sustained retention, positive attitude, persuasive arguments, and well-formatted assignments will earn the most credit.
    \end{description}

\section{Course Policies}

\subsection{In-Class Confidentiality}
The success of this class depends on participants feeling comfortable sharing questions, ideas, concerns, and confusions about works-in-progress, the research process, and their personal experiences. These assignments and discussions should be considered confidential and generally should not be discussed outside of class. You may read and comment on classmates' writing, code, and images for the sole purpose of use within this class. You may not use, run, copy, perform, display, distribute, modify, translate, or create derivative works of other students' work outside of this class without their expressed written consent or formal license. Furthermore, you may not create any audio, video, or other records during class time nor may you publicly share comments attributable to other people's identities without that person's permission.

\subsection{Faculty Interaction}
In addition to teaching this class, I also (1) manage a research program; (2) advise students; (3) perform service for the academic community; and (4) live my life as a private citizen. I will check e-mail between 8:00 and 18:00 on non-holiday business days and try to respond within 24 hours. I welcome online or offline interactions outside of class, however these are not appropriate spaces for discussing class matters. \href{maito:brian.keegan@colorado.edu}{E-mailing me}, coming to my office hours, or scheduling an appointment are the best ways to ask questions, discuss concerns, or get feedback outside of class.

%\subsection{Research Ethics and Professional Conduct}
%This course will involve participating in online communities, analyzing data available on the web, and prototyping alternative peer production and crowdsourcing systems. At all times you should make sure that your actions simultaneously minimize the risk of harm to other people as well to yourself. You should follow the Terms of Service and Privacy Policy when collecting data from websites; only collect publicly-accessible data; identify yourself as a student or researcher when interacting with others; and allow participants to remove themselves from your study if they request it. More details about research ethics with online data can be found in the Association for Computing Machinery's \href{https://www.acm.org/about-acm/acm-code-of-ethics-and-professional-conduct}{Code of Ethics and Professional Conduct}, Association for Internet Researchers' \href{http://aoir.org/ethics/}{ethics working committee reports}, American Psychological Association's \href{http://www.apa.org/science/leadership/bsa/internet/internet-report.aspx}{online research report}.

\subsection{Deadlines and Absences}
If something causes you to miss a deadline or a class, please contact me. If you request --- and obtain --- an incomplete for the course and/or an extension on the final project (note: I \textit{strongly} discourage this!), please allow at least 4 weeks after you submit your completed work for me to submit a grade. Keep this in mind if you will need the grade to receive your fellowship/diploma/visa/\textit{etc}. by a particular date.

\subsection{Accommodations for Disabilities}
I am committed to providing everyone the support and services needed to participate in this course. If you qualify for accommodations because of a disability, please submit to your professor a letter from Disability Services in a timely manner (for exam accommodations provide your letter at least one week prior to the exam) so that your needs can be addressed. Disability Services determines accommodations based on documented disabilities. Contact Disability Services at 303-492-8671 or by e-mail at \href{mailto:dsinfo@colorado.edu}{dsinfo@colorado.edu}. If you have a temporary medical condition or injury, see \href{http://www.colorado.edu/disabilityservices/students/temporary-medical-conditions}{Temporary Medical Conditions} guidelines under Quick Links at \href{http://www.colorado.edu/disabilityservices/students}{Disability Services} website and discuss your needs with me.

\subsection{Religious Observance}
Campus policy regarding \href{http://www.colorado.edu/policies/observance-religious-holidays-and-absences-classes-andor-exams}{religious observances} requires that faculty make every effort to deal reasonably and fairly with all students who, because of religious obligations, have conflicts with scheduled exams, assignments or required assignments/attendance. If this applies to you, please email me directly as soon as possible at the beginning of the term. 

\subsection{Classroom Behavior}
Students and faculty each have responsibility for maintaining an appropriate learning environment. Those who fail to adhere to such behavioral standards may be subject to discipline. Professional courtesy and sensitivity are especially important with respect to individuals and topics dealing with differences of race, color, culture, religion, creed, politics, veteran’s status, sexual orientation, gender, gender identity and gender expression, age, ability, and nationality.  Class rosters are provided to the instructor with the student's legal name. I will gladly honor your request to address you by an alternate name or gender pronoun. Please advise me of this preference early in the semester so that I may make appropriate changes. For more information, see the policies on \href{http://www.colorado.edu/policies/student-classroom-and-course-related-behavior}{class behavior} and the \href{http://www.colorado.edu/osc/#student_code}{student code}.

\subsection{Harassment and Discrimination}
The University of Colorado Boulder (CU Boulder) is committed to maintaining a positive learning, working, and living environment. CU Boulder will not tolerate acts of sexual misconduct, discrimination, harassment or related retaliation against or by any employee or student. CU's \href{http://www.colorado.edu/policies/discrimination-and-harassment-policy-and-procedures}{Sexual Misconduct Policy} prohibits sexual assault, sexual exploitation, sexual harassment, intimate partner abuse (dating or domestic violence), stalking or related retaliation. CU Boulder's \href{http://www.colorado.edu/policies/discrimination-and-harassment-policy-and-procedures}{Discrimination and Harassment Policy} prohibits discrimination, harassment or related retaliation based on race, color, national origin, sex, pregnancy, age, disability, creed, religion, sexual orientation, gender identity, gender expression, veteran status, political affiliation or political philosophy. Individuals who believe they have been subject to misconduct under either policy should contact the Office of Institutional Equity and Compliance (OIEC) at 303-492-2127. Information about the OIEC, the above referenced policies, and the campus resources available to assist individuals regarding sexual misconduct, discrimination, harassment or related retaliation can be found at the \href{http://www.colorado.edu/institutionalequity/}{OIEC website}.

\subsection{Honor Code}
All students enrolled in a University of Colorado Boulder course are responsible for knowing and adhering to the \href{http://www.colorado.edu/policies/academic-integrity-policy}{academic integrity policy} of the institution. Violations of the policy may include: plagiarism, cheating, fabrication, lying, bribery, threat, unauthorized access, clicker fraud, resubmission, and aiding academic dishonesty. All incidents of academic misconduct will be reported to the Honor Code Council (\href{mailto:honor@colorado.edu}{honor@colorado.edu}; 303-735-2273). Students who are found responsible for violating the academic integrity policy will be subject to nonacademic sanctions from the Honor Code Council as well as academic sanctions from the faculty member. Additional information regarding the academic integrity policy can be found at \href{http://honorcode.colorado.edu}{honorcode.colorado.edu}. 

\section{Acknowledgements}
The design and format of this course borrows from similar courses and textbooks. You are not required to buy these books or classes, but these can be very helpful resources if you find yourself struggling.
\begin{itemize}[itemsep=1em]
    \item Textbooks
        \begin{itemize}
        \item \bibentry{bhargava_grokking_2016}
        \item \bibentry{goodrich_data_2013}
        \item \bibentry{heineman_algorithms_2016}
        \item \bibentry{hetland_python_2014}
        \item \bibentry{karumanchi_data_2015}
        \item \bibentry{lee_data_2015}
        \item \bibentry{miller_problem_2013}
        \end{itemize}
    \item Online courses
        \begin{itemize}
        \item \bibentry{balazs_algorithms_2016}
        \item \bibentry{kane_algorithmic_2016}
        \item \bibentry{nakhleh_algorithmic1_2016}
        \item \bibentry{portilla_python_2016}
        \item \bibentry{roughgarden_online_2016}
        \item \bibentry{sentance_python_2016}
        \end{itemize}
\end{itemize}


%%%%%%%%%%%%%%%%%%%%%%
%%%%%%%%%%%%%%%%%%%%%%
%%% COURSE OUTLINE %%%
%%%%%%%%%%%%%%%%%%%%%%
%%%%%%%%%%%%%%%%%%%%%%

\newpage
\section{\textbf{Course Outline}}

I may adjust the list of readings or the schedule as needed throughout the semester, so please consult the schedule online at Desire2Learn for the most up-to-date information. We will meet in class a total of 44 times over the course of the the 16-week semester.

% Confirm research computing environments for Python and \texttt{R} are properly configured and ready for data collection and analysis. 

\section{Week 1 -- Structures: Python Fundamentals}
\textcolor{CUGold}{\textbf{Wednesday, January 18; Friday, January 20}}\\
Configure computing environment and review operations on fundamental Python data structures.
    
    \subsection{Objectives}
    \begin{itemize}
        \item Introduce Jupyter Notebook and SageMathCloud environments.
        \item Review basic Python syntax and documentation standards.
        \item Review fundamental data structures in Python: lists, dictionaries, tuples, and objects.
    \end{itemize}
    
    \subsection{Assignments}
    \begin{description}%[itemsep=2pt,labelsep=0pt]
        \item[Lab 1 ] -- due Wednesday, January 25. 
    \end{description}

\section{Week 2 -- Structures: Sequences}
\textcolor{CUGold}{\textbf{Monday, January 23; Wednesday, January 25; Friday, January 27}}\\
Introduction to linear data structures like arrays, linked lists, and queues.
    
    \subsection{Objectives}
    \begin{itemize}
        \item 
        \item 
        \item 
    \end{itemize}

    \subsection{Assignments}
    \begin{description}%[itemsep=2pt,labelsep=0pt]
        \item[Lab 2 ] -- due Wednesday, February 1
    \end{description}

\section{Week 3 -- Structures: Maps}
\textcolor{CUGold}{\textbf{Monday, January 30; Wednesday, February 1; Friday, February 3}}\\
Introduction to data collections where values are indexed by unique keys.

    \subsection{Objectives}
    \begin{itemize}
        \item 
        \item 
        \item 
    \end{itemize}

    \subsection{Assignments}
    \begin{description}%[itemsep=2pt,labelsep=0pt]
        \item[Lab 3 ] -- due Wednesday, February 8.
    \end{description}

\section{Week 4 -- Structures: Recursion and Complexity}
\textcolor{CUGold}{\textbf{Monday, February 6; Wednesday, February 8; Friday, February 10}}\\
Developing intuitions using recursive functions and measuring the time-complexity of algorithms.

    \subsection{Objectives}
    \begin{itemize}
        \item 
        \item 
        \item 
    \end{itemize}

    \subsection{Assignments}
    \begin{description}%[itemsep=2pt,labelsep=0pt]
        \item[Lab 4 ] -- due Wednesday, February 15.
    \end{description}

\section{Week 5 -- Algorithms: Searching and Sorting}
\textcolor{CUGold}{\textbf{Monday, February 13; Wednesday, February 15; Friday, February 17}}\\
Organizing data by efficiently finding and sorting values.

    \subsection{Objectives}
    \begin{itemize}
        \item 
        \item 
        \item 
    \end{itemize}

    \subsection{Assignments}
    \begin{description}%[itemsep=2pt,labelsep=0pt]
        \item[Lab 5 ] -- due Wednesday, February 22.
    \end{description}

\section{Week 6 -- Algorithms: Comparisons and Distances}
\textcolor{CUGold}{\textbf{Monday, February 20; Wednesday, February 22; Friday, February 24}}\\
Comparing strings and computing distances between data values.

    \subsection{Objectives}
    \begin{itemize}
        \item 
        \item 
        \item 
    \end{itemize}

    \subsection{Assignments}
    \begin{description}%[itemsep=2pt,labelsep=0pt]
        \item[Lab 6 ] -- due Wednesday, March 1.
    \end{description}
    
\section{Week 7 -- Algorithms: Trees}
\textcolor{CUGold}{\textbf{Monday, February 27; Wednesday, March 1; Friday, March 3}}\\
Accessing and analyzing hierarchical properties of trees. 

\textbf{NOTE:} \textit{Professor Keegan will be attending the \href{https://cscw.acm.org/2017/}{CSCW 2017} conference Monday and Wednesday.}

    \subsection{Objectives}
    \begin{itemize}
        \item 
        \item 
        \item 
    \end{itemize}

    \subsection{Assignments}
    \begin{description}%[itemsep=2pt,labelsep=0pt]
        \item[Lab 7 ] -- due Wednesday, March 8.
    \end{description}
    
\section{Week 8 -- Algorithms: Graphs}
\textcolor{CUGold}{\textbf{Monday, March 6; Wednesday, March 8; Friday, March 10}}\\
Accessing and analyzing structural properties of graphs and complex networks.

    \subsection{Objectives}
    \begin{itemize}
        \item 
        \item 
        \item 
    \end{itemize}

    \subsection{Assignments}
    \begin{description}%[itemsep=2pt,labelsep=0pt]
        \item[Lab 8 ] -- due Wednesday, March 15.
    \end{description}
    
\section{Week 9 -- Shapes: Long and Wide}
\textcolor{CUGold}{\textbf{Monday, March 13; Wednesday, March 15; Friday, March 17}}\\
Reshaping ``wide'' data into ``long'' data and back using melting and stacking operations.

    \subsection{Objectives}
    \begin{itemize}
        \item 
        \item 
        \item 
    \end{itemize}

    \subsection{Assignments}
    \begin{description}%[itemsep=2pt,labelsep=0pt]
        \item[Lab 9 ] -- due Wednesday, March 22.
    \end{description}
    
\section{Week 10 -- Shapes: Arrays and Hierarchical Indices}
\textcolor{CUGold}{\textbf{Monday, March 20; Wednesday, March 22; Friday, March 24}}\\
Accessing and managing multidimensional data structures in \texttt{numpy} and \texttt{pandas}.

    \subsection{Objectives}
    \begin{itemize}
        \item 
        \item 
        \item 
    \end{itemize}

    \subsection{Assignments}
    \begin{description}%[itemsep=2pt,labelsep=0pt]
        \item[Lab 10 ] -- due Wednesday, April 5.
    \end{description}
    
\section{Week 11 -- Spring Break}
\textcolor{CUGold}{\textbf{Monday, March 27; Wednesday, March 29; Friday, March 31}}\\
Adding new objects into long-term memory.

\section{Week 12 -- Shapes: Markup Languages}
\textcolor{CUGold}{\textbf{Monday, April 3; Wednesday, April 5; Friday, April 7}}\\
Accessing and reshaping markup languages like XML and JSON.

    \subsection{Objectives}
    \begin{itemize}
        \item 
        \item 
        \item 
    \end{itemize}

    \subsection{Assignments}
    \begin{description}%[itemsep=2pt,labelsep=0pt]
        \item[Lab 12 ] -- due Wednesday, April 12.
    \end{description}
    
\section{Week 13 -- Queries: Counting}
\textcolor{CUGold}{\textbf{Monday, April 10; Wednesday April 12; Friday, April 14}}\\
Introduction to basic SQL syntax for selecting, filtering, sorting, and counting data.

    \subsection{Objectives}
    \begin{itemize}
        \item 
        \item 
        \item 
    \end{itemize}

    \subsection{Assignments}
    \begin{description}%[itemsep=2pt,labelsep=0pt]
        \item[Lab 13 ] -- due Wednesday, April 19.
    \end{description}
    
\section{Week 14 -- Queries: Joining}
\textcolor{CUGold}{\textbf{Monday, April 17; Wednesday, April 19; Friday, April 21}}\\
Joining data from two different tables together and dealing with missing or repeated data. 

    \subsection{Objectives}
    \begin{itemize}
        \item 
        \item 
        \item 
    \end{itemize}

    \subsection{Assignments}
    \begin{description}%[itemsep=2pt,labelsep=0pt]
        \item[Lab 14 ] -- due Wednesday, April 26.
    \end{description}
    
\section{Week 15 -- Queries: Grouping}
\textcolor{CUGold}{\textbf{Monday, April 24; Wednesday, April 26; Friday, April 28}}\\
Grouping repeated observations together and computing their aggregated values.

    \subsection{Objectives}
    \begin{itemize}
        \item 
        \item 
        \item 
    \end{itemize}

    \subsection{Assignments}
    \begin{description}%[itemsep=2pt,labelsep=0pt]
        \item[Lab 15 ] -- due Wednesday, May 3.
    \end{description}
    
\section{Week 16 -- Queries: Non-Relational Databases}
\textcolor{CUGold}{\textbf{Monday, May 1; Wednesday, May 3; Friday, May 5}}\\
NoSQL and graph databases as alternatives to relational databases.

    \subsection{Objectives}
    \begin{itemize}
        \item 
        \item 
        \item 
    \end{itemize}

    \subsection{Assignments}
    \begin{description}%[itemsep=2pt,labelsep=0pt]
        \item[Lab 16 ] -- due Wednesday, May 12.
    \end{description}

% Include summary table
\afterpage{%
    \clearpage
    \thispagestyle{empty}
    \begin{landscape}
        \centering
        \Large
        \begin{tabular}{ccccc}
            \toprule[.15em]
            \textbf{Module} & \textbf{Week} & \textbf{Dates} & \textbf{Theme} & \textbf{Topics} \\
            \cmidrule[.1em](lr){1-5}
            \multirow{5}{*}[0pt]{\textit{Structures}}
                & 1 & Jan 18 -- Jan 20 & Fundamentals and Review & lists, dictionaries, tuples, objects \\
                & 2 & Jan 23 -- Jan 27 & Recursion and Complexity & flow control, big-O \\
                & 3 & Jan 30 -- Feb 3 & Arrays and Lists & lists, linked lists, matrices \\
                & 4 & Feb 6 -- Feb 10 & Queues and Stacks &  \\
                & 5 & Feb 13 -- Feb 17 & Sets and Maps & sets, hashes, dictionaries \\ \cmidrule[.1em](lr){1-5}
            \multirow{5}{*}[0pt]{\textit{Algorithms}} 
                
                & 6 & Feb 20 -- Feb 24 & Strings & similarity, searching, matching \\
                & 7 & Feb 27 -- Mar 3 & Searching & linear, binary, interpolation \\
                & 8 & Mar 6 -- Mar 10 & Sorting & insertion, bubble, merge \\
                & 9 & Mar 13 -- Mar 17 & Trees & representations, operations, traversal  \\ 
                & 10 & Mar 20 -- Mar 24 & Graphs & directed/undirected, depth/breadth search \\ \cmidrule[.1em](lr){1-5}
                & 11 & Mar 27 -- Mar 31 & \multicolumn{2}{c}{Spring Break} \\
                \cmidrule[.1em](lr){1-5}
            \multirow{5}{*}[0pt]{\textit{Queries}} 
                & 12 & Apr 3 -- Apr 7 & Shapes and Markup & wide/long data, accessing JSON/XML \\
                & 13 & Apr 10 -- Apr 14 & Counting & selecting, filtering, sorting \\
                & 14 & Apr 17 -- Apr 21 & Joining & inner, outer, left, right joins \\
                & 15 & Apr 24 -- Apr 28 & Grouping & aggregation functions, having clause \\
                & 16 & May 1 -- May 5 & Non-Relational Databases & NoSQL and graph databases \\
            \bottomrule[.15em]
        \end{tabular}\\
        %\caption{Course Schedule}% Add 'table' caption
    \end{landscape}
    \clearpage% Flush page
}

% bibliography here
%\newpage
\renewcommand{\bibsection}{\section{\huge \bibname}\prebibhook}
\baselineskip 14.2pt
\nobibliography{refs}
\bibliographystyle{apalike}

\end{document}
